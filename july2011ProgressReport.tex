\documentclass[a4paper]{article}
\title{A Progress Report of the Manchester University ARM Port of Dynamo-RIO}
\author{James Sandford\\
School of Computer Science,\\
The University of Manchester,\\
Oxford Road,\\
Manchester,\\
M13 9PL
\texttt{sandfoj9@cs.man.ac.uk}}
\date{\today}

\usepackage{url}

\begin{document}


\maketitle
\newpage


\begin{abstract}
This report has been put together following several weeks of trying to
understand the Dynamo-RIO software project and the current progress made in it's
port to the ARM architecture from the original x86 code carried out originally 
by Stephen Barton as a final year computer science project in the year prior to 
this report. I aim to provide a means to get to my current possition as quickly
as possible by showing how to get the code cross compiling on an x86 Linux
system, pitfalls I fell down when going through this process and others, an
overview of the code structure and an overview of what has and has not been
completed in the port at this point in time. In addition to this, I will provide
any advice I can on working with the current code base and I will provide
recomendations on how to progress from here.
\end{abstract}
\newpage


\tableofcontents
\newpage


\section{The code base}
\subsection{Git and GitHub}
The code for the ARM port is currently available on GitHub and can be found at
\url{http://github.com/j616/DynamoRIO-for-ARM/}. I shalln't go into any detail
on how to use Git or GitHub here but you can find all you need to know at
\url{http://help.github.com/}. But Git is basically a version control system
that's a bit nicer when it comes to things such as merging than systems like
SVN. GitHub is just a hosting website for Git repositories that provides a nice
web interface with some occasionally usefull features such as graphing of
contribution and programming languages along with the usual browser based
viewing and editing of files.

\subsection{The Folder Structure}
The folder structure consists of the following:
\begin{description}
\item[Clients] Several example clients.
\item[CurrentVersion] The copy of the source being worked on. This is our main
focus.
	\begin{description}
	\item[api] Strangely named as it contains the standard Dynamo-RIO sample clients
	such as bbcount.
	\item[bin] The shell scripts to run Dynamo-RIO. For information on these, look
	at the README in the CurrentVersion folder.
	\item[clients] A couple of clients. These are Windows specific so I haven't
	looked at them. I should note here that the ARM port is currently Linux specific
	as Windows doesn't currently support ARM. Though I believe this will change with
	Windows 8.
	\item[cmake] CMake configuration files. Don't really need touching but hold
	things like the version number to give your build.
	\item[core] The place where things get interesting. This is Dynamo-RIO itself in
	many ways. This is the code that manages everything else. Importantly, this is
	where we maintain control of the software we are working on along with the
	decoding, modification and encoding of instructions. Most (all?) of the ARM
	specific code will be in here.
	\begin{description}
		\item[Arm] The (mainly) ARM specific files. More on this later.
		\item[lib] The header files and generation scripts for the client libraries.
		\item[linux] The Linux specific files.
		\item[win32] The Windows specific files (not just 32bit).
		\item[x86] The (mainly) x86 specific files. More on this later.
	\end{description}
	\item[ext] Extentions. I believe this is needed for the actual use of clients in
	Dynamo-RIO.
	\item[include] The basic header files you will use when constructing clients.
	There is a major problem here that will be discussed later.
	\item[libutil] General libraries. I haven't needed to touch these.
	\item[make] Auto-generated make files.
	\item[suite] A suite of development tools for the main Dynamo-RIO devs. This
	hasn't and probably shouldn't be used in the ARM build at least while it remains
	a seperate project.
	\item[tools] Again, a large number of development tools. These again look mainly
	related to the main version of Dynamo-RIO and to x86. Some may be adaptable
	though for debuging and testing.
	\end{description}
\item[Decoder] A standalone ARM machine code decoder. This will be explained
more later.
\item[MiscCode] Samples of ARM assembly, x86 assembly, very simple test C
programs for running Dynamo-RIO on and cMake script examples.
\begin{description}
	\item[ArmAssembly] Some small test programs in C and ARM assembly. More on
	this later.
	\item[Assembly] Some small test programs in x86 assembly.
	\item[cMakeExamples] Some small example scripts for cMake.
\end{description}
\item[RunningVersion] A copy of the source containing mainly x86 specific code.
This has remained largely un-used by myself and thus I won't go into any detail.
Best sticking with CurrentVersion.
\end{description}





\end{document}
